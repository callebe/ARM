\section{Pré-requisitos}
Para que se possa usufruir de todos os conceitos e práticas presentes neste material, se faz necessário um conhecimento prévio, por parte do leitor, sobre as áreas de microcontrolados e arquitetura de computadores. Bem como uma base em programação, especialmente na linguagem C.

\section{Orientações Pedagógicas}

Os capítulos 2 e 3 se destinam a introduzir os conhecimentos sobre a arquitetura, plataforma e bibliotecas a serem utilizadas nos demais capítulos, servindo como prelúdio. O capítulo 2  destina-se a introduzir o processador ARM, explicando inicialmente suas principais características, modos de operação e registradores internos. Já o capitulo 3 apresenta um resumo sobre os principais periféricos presentes na plataforma de trabalho, o Tiva\textsuperscript{TM} TM4C1294NCPDT.

Em seguida têm-se os capítulos 4 e 5, que apresentam a IDE e as bibliotecas de \emph{software} necessárias e o procedimento para utilizá-las. O capítulo 4 guia o leitor na tarefa de  iniciar um novo projeto na IDE \emph{Code Composer}. Para que se possa utilizar tais bibliotecas de \emph{software}, o capítulo 5 se encarrega de mostrar o passo a passo de como incluí-las ao novo projeto.

Já os capítulos 6 ao 12 sempre seguem o roteiro de apresentar uma introdução teórica sobre o periférico, em seguida demonstrar como este funciona no Tiva\textsuperscript{TM} TM4C1294NCPDT e como é feita sua implementação através das bibliotecas de \emph{software}. Ao fim de cada capítulo, há um exemplo prático de configuração de cada periférico. 

Ao final do material, no capítulo 12, são apresentados exemplos de aplicações utilizando os  periféricos demonstrados ao longo dos capítulos anteriores.  