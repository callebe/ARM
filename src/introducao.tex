O material aqui contido busca iniciar o leitor no desenvolvimento em plataformas baseadas em processadores com arquitetura ARM.

Inicialmente, o texto introduz alguns conceitos sobre a arquitetura ARM, em especial as Cortex-M3 e M4. São discutidos seus registradores padrão e seus modos de operação. Após isso, são abordadas as características das plataformas específicas utilizadas nos exemplos deste material. Os \emph{hardwares} utilizados aqui, são o kit de 
avaliação Tiva\textsuperscript{TM} C Series TM4C1294NCPDT, e o MSP432P401R. Ambos da empresa Texas Instruments.

Logo após, são introduzidos os padrões utilizados neste material: a biblioteca TivaWare e o padrão CMSIS. Ainda, são mostrados métodos de utilizá-los a partir da IDE da Texas Instruments, o Code Composer Studio.

Os capítulos centrais abordam algumas características do TM4C1294NCPDT, e ainda, como são implementadas através da TivaWare. São abordados a maioria dos periféricos principais, sendo que cada capítulo aborda um deles em especial.

Ao final, são apresentados exemplos práticos de implementação dos principais periféricos. Tanto utilizando a TivaWare, para o TM4C1294NCPDT, quanto utilizando a CMSIS, para o MSP432P401R.
