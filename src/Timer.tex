
Temporizadores em microcontroladores são usados como contadores de intervalo de tempo para eventos internos e externos ao DSP. Tais temporizadores possibilitam provocar interrupções a tempos programáveis, sendo esse o tipo de recurso essencial em muitas aplicações.  

\section{GPTM no TM4C1294NCPDT}

O temporizador de propósito geral, ou \emph{General-Purpose Timers} (GPTM),  do Tiva - TM4C129NCPDT possui blocos de contadores de 16/32 bits. Em cada bloco há dois contadores de 16 bits, sendo um do \emph{Timer A} e o outro do \emph{Timer B}, que podem ser concatenados em apenas um contador de 32 bits.

Qualquer temporizador do Tiva é capaz de ser usado como um gatilho para conversões do ADC. Porém não se pode usar mais de um temporizador como gatilho para o ADC.

O GPTM é o recurso de temporizador padrão no Tiva, porém ele não é o único temporizador presente. Existem os temporizadores do módulo PWM, do módulo \emph{WatchDog Timer}, e do módulo \emph{Systick}, que também podem ser usados de modo parecido ao GPTM. 

Dentro do GPTM há 8 blocos de contadores de 16/32 bits. A fonte de clock para os contadores pode ser selecionada entre o clock do sistema ou o ALTCLK (\emph{Alternate CLock}). Tendo em vista que o ALTCLK pode ser proveniente do PIOSC, clock do módulo de hibernação ou o oscilador de baixa frequência (LFIOSC).  As funcionalidades básicas dos contadores de cada bloco são: 

\begin{itemize}
	\item Provocar uma única interrupção após intervalo de tempo programável, com contador de 16/32 bits.
	\item Provocar interrupções periódicas a intervalo de tempo programável, com contador de 16/32 bits.
	\item Clock de tempo real utilizando clock externo de 32,768 kHz, com contador de 32 bits
	\item Captura de intervalo de tempo através de detecção de borda de sinal externo com divisor de clock de 8 bits, com contador de 16 bits. 
	\item Modo PWM, com divisor de clock de 8 bits, com contador de 16 bits.
	\item Contador do modo contagem crescente ou decrescente.
	\item Modo de captura e comparação, com contador de 16/32 bits.
	\item Encadeamento de temporizadores para múltiplos disparos de interrupção.
	\item Gatilho para disparo do ADC.
\end{itemize}

A Tabela \ref{tab:CanaisTimer} apresenta os pinos de entrada e saída do GPTM para os modos de funcionamento captura, comparação e geração de PWM.

\begin{center}
	\begin{longtable}{|c|c|c|c|c|}
		\rowcolor[HTML]{000000}
		{\color[HTML]{FFFFFF} Pino} & {\color[HTML]{FFFFFF} $n^{o}$} & {\color[HTML]{FFFFFF} Mux/Função} & {\color[HTML]{FFFFFF} Tipo} & {\color[HTML]{FFFFFF} Descrição}            \\
		\multirow{3}{*}{T0CCP0}    & 1   & PD0 (3) & \multirow{3}{*}{I/O} & \multirow{3}{*}{Timer 0 Captura/Comparação/PWM 0}\\
		& 33  & PA0 (3) &     &                                 \\
		& 85  & PL4 (3) &     &                                 \\ \hline
		\multirow{3}{*}{T0CCP1}    & 2   & PA1 (6) & \multirow{3}{*}{I/O} & \multirow{3}{*}{Timer 0 Captura/Comparação/PWM 1}\\
		& 34  & PL5 (3) &     &                                 \\
		& 86  & PL5 (3) &     &                                 \\ \hline
		\multirow{3}{*}{T1CCP0}    & 3   & PD2 (3) & \multirow{3}{*}{I/O} & \multirow{3}{*}{Timer 1 Captura/Comparação/PWM 0}\\
		& 35  & PA2 (3) &     &                                 \\
		& 94  & PL6 (3) &     &                                 \\ \hline
		\multirow{3}{*}{T1CCP1}    & 4   & PD3 (3) & \multirow{3}{*}{I/O} & \multirow{3}{*}{Timer 1 Captura/Comparação/PWM 1}\\
		& 36  & PA3 (3) &     &                                 \\
		& 93  & PL7 (3) &     &                                 \\ \hline
		\multirow{2}{*}{T2CCP0}   & 37  & PA4 (3) & \multirow{2}{*}{I/O} & \multirow{2}{*}{Timer 2 Captura/Comparação/PWM 0}\\
		& 78  & PM0 (3) &     &                                 \\ \hline
		\multirow{2}{*}{T2CCP1}    & 38  & PA5 (3) & \multirow{2}{*}{I/O} & \multirow{2}{*}{Timer 2 Captura/Comparação/PWM 1}\\
		& 77  & PM1 (3) &     &                                 \\ \hline
		\multirow{3}{*}{T3CCP0}    & 40  & PA6 (3) & \multirow{3}{*}{I/O} & \multirow{3}{*}{Timer 3 Captura/Comparação/PWM 0}\\
		& 76  & PM2 (3) &     &                                 \\ 
		& 125 & PD4 (3) &     &                                 \\ \hline
		\multirow{3}{*}{T3CCP1}    & 41  & PA7 (3) & \multirow{3}{*}{I/O} & \multirow{3}{*}{Timer 3 Captura/Comparação/PWM 1}\\
		& 75  & PM3 (3) &     &                                 \\ 
		& 126 & PD5 (3) &     &                                 \\ \hline
		\multirow{3}{*}{T4CCP0}    & 74  & PM4 (3) & \multirow{3}{*}{I/O} & \multirow{3}{*}{Timer 4 Captura/Comparação/PWM 0}\\
		& 95  & PB0 (3) &     &                                 \\ 
		& 127 & PD6 (3) &     &                                 \\ \hline
		\multirow{3}{*}{T4CCP1}    & 73  & PM5 (3) & \multirow{3}{*}{I/O} & \multirow{3}{*}{Timer 4 Captura/Comparação/PWM 1}\\
		& 96  & PB1 (3) &     &                                 \\ 
		& 128 & PD7 (3) &     &                                 \\ \hline
		\multirow{2}{*}{T5CCP0}    & 72  & PM6 (3) & \multirow{2}{*}{I/O} & \multirow{2}{*}{Timer 5 Captura/Comparação/PWM 0}\\
		& 91  & PB2 (3) &     &                                 \\ \hline
		\multirow{2}{*}{T5CCP1}    & 71  & PM7 (3) & \multirow{2}{*}{I/O} & \multirow{2}{*}{Timer 5 Captura/Comparação/PWM 1}\\
		& 92  & PB3 (3) &     &                                 \\ \hline
		\caption{Canais Temporizador de Propósito Geral - Tiva TM4C1294NCPDT \cite{DATASHEET_TIVA} }
		\label{tab:CanaisTimer}
	\end{longtable}
\end{center}

\section{GPTM na TivaWare}

\section{Exemplo}