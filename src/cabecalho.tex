\usepackage[utf8]{inputenc}
\usepackage{float}
%Uma das dificuldades para os usuários do Latex é o posicionamento das figuras, já que ele tenta fazer isso de forma otimizada e automática. Se o usuário quizer forçar as figuras existem duas possibilidades:(1) Definir o posicionamento em: (h) aqui, (t) topo, (b) base, ou na (p) página. O uso do ! siginifica para forçar a solicitação do usuário. a ordem [htb] indica a preferencia que o latex vai re-arranjar as figuras.%(2%) Muitas vezes o usuário não quer otimizar o espaço do texto. Ou seja que posicionar a figura em determinado lugar mesmo que um pedaço da págian fique em branco. Neste caso o indicado é usar o pacote: float e o posicionamento H.%
\usepackage{hyperref}
\usepackage{graphicx}
\usepackage{textcomp}
\usepackage{gensymb}
\usepackage[table,xcdraw]{xcolor}
\usepackage{listings}
\usepackage{cmap}
\usepackage[T1]{fontenc}
\usepackage{verbatim}
\usepackage[table,xcdraw]{xcolor}
\usepackage{longtable}
\usepackage{multirow}
\usepackage[affil-it]{authblk}
\usepackage{multicol}

\usepackage[brazil]{minitoc}
\mtcsettitle{minitoc}{Lista de Exemplos}
\newcommand{\filterminitoc}[1]{#1}
\renewcommand{\thesection}{\csname filterminitoc \endcsname{\arabic{chapter}.}\arabic{section}}
\newcommand{\minitocsection}{\begingroup\renewcommand{\filterminitoc}[1]{}\minitoc\endgroup}

% Configurando layout para mostrar codigos C++
\usepackage{listings}
% Definindo novas cores
\usepackage{color}

\usepackage{geometry}
	\geometry{
	a4paper,
	total={170mm,257mm},
	left=40mm,
	right=30mm,
	top=40mm,
	bottom=30mm,
 }

\definecolor{mygreen}{rgb}{0,0.6,0}
\definecolor{mygray}{rgb}{0.5,0.5,0.5}
\definecolor{black}{rgb}{0.3,0.3,0.3}
\definecolor{gray}{rgb}{0.9,0.9,1}
\definecolor{mymauve}{rgb}{0.58,0,0.82}
\lstset{ %
  backgroundcolor=\color{white},  
  basicstyle=\ttfamily\small,            
  breakatwhitespace=false,         
  breaklines=true,                 
  captionpos=b,                    
  commentstyle=\color{mygreen},  
  deletekeywords={...},            
  escapeinside={\%*}{*)},  
  extendedchars=true,    
  %frame=single,	                  
  keepspaces=true,                 
  keywordstyle=\color{blue},       
  language=Octave,,                 
  otherkeywords={*,...},          
  numbers=left,                   
  numbersep=5pt,                   
  numberstyle=\ttfamily\small\color{mygray},
  rulecolor=\color{black},         
  showspaces=false,                
  showstringspaces=false,          
  showtabs=false,                  
  stepnumber=2,                    
  stringstyle=\color{mymauve},  
  tabsize=2,	                   
  title=\lstname,
}


\lstdefinestyle{citacao}{ %
  backgroundcolor=\color{white},   % choose the background color; you must add \usepackage{color} or \usepackage{xcolor}
  basicstyle=\ttfamily,        % the size of the fonts that are used for the code
  breakatwhitespace=false,         % sets if automatic breaks should only happen at whitespace
  breakautoindent=true,
  %breaklines=true,                 % sets automatic line breaking
  columns=fixed,			% don't add spaces between characters
  commentstyle=\color{mygreen},    % comment style
  deletekeywords={...},            % if you want to delete keywords from the given language
  escapeinside={\%*}{*)},          % if you want to add LaTeX within your code
  extendedchars=true,              % lets you use non-ASCII characters; for 8-bits encodings only, does not work with UTF-8
  frame=simple,	                   % adds a frame around the code
  keepspaces=true,                 % keeps spaces in text, useful for keeping indentation of code (possibly needs columns=flexible)
  keywordstyle= \color{blue}\bf,       % keyword style
  language=C,                 % the language of the code
  otherkeywords={},           % if you want to add more keywords to the set
  numbers=left,                    % where to put the line-numbers; possible values are (none, left, right)
  numbersep=5pt,                   % how far the line-numbers are from the code
  numberstyle=\normalfont\color{mygray}, % the style that is used for the line-numbers
  rulecolor=\color{black},         % if not set, the frame-color may be changed on line-breaks within not-black text (e.g. comments (green here))
  showspaces=false,                % show spaces everywhere adding particular underscores; it overrides 'showstringspaces'
  showstringspaces=false,          % underline spaces within strings only
  showtabs=false,                  % show tabs within strings adding particular underscores
  stepnumber=1,                    % the step between two line-numbers. If it's 1, each line will be numbered
  stringstyle=\color{mymauve},     % string literal style
  tabsize=4,	                   % sets default tabsize to 2 spaces
  title=\lstname                   % show the filename of files included with \lstinputlisting; also try caption instead of title
}

\lstdefinestyle{funcao}{ %
  backgroundcolor=\color{white},   % choose the background color; you must add \usepackage{color} or \usepackage{xcolor}
  basicstyle=\ttfamily,        % the size of the fonts that are used for the code
  breakatwhitespace=false,         % sets if automatic breaks should only happen at whitespace
  breakautoindent=true,
  %breaklines=true,                 % sets automatic line breaking
  columns=fixed,			% don't add spaces between characters
  commentstyle=\color{mygreen},    % comment style
  deletekeywords={...},            % if you want to delete keywords from the given language
  escapeinside={\%*}{*)},          % if you want to add LaTeX within your code
  extendedchars=true,              % lets you use non-ASCII characters; for 8-bits encodings only, does not work with UTF-8
  frame=trBL,	                   % adds a frame around the code
  frameround=T,
  rulesep=1pt,
  framesep=5pt,
  rulesepcolor=\color{gray},
  framexleftmargin=-20pt,
  xleftmargin=-20pt,
  xrightmargin=0pt,
  keepspaces=true,                 % keeps spaces in text, useful for keeping indentation of code (possibly needs columns=flexible)
  keywordstyle=\normalcolor\normalfont\normalsize,       % keyword style
  language=C,                 % the language of the code
  otherkeywords={},           % if you want to add more keywords to the set
  numbers=none,                    % where to put the line-numbers; possible values are (none, left, right)
  rulecolor=\color{black},         % if not set, the frame-color may be changed on line-breaks within not-black text (e.g. comments (green here))
  showspaces=false,                % show spaces everywhere adding particular underscores; it overrides 'showstringspaces'
  showstringspaces=false,          % underline spaces within strings only
  showtabs=false,                  % show tabs within strings adding particular underscores
  stringstyle=\color{mymauve},     % string literal style
  tabsize=4,	                   % sets default tabsize to 2 spaces
  title=\lstname                   % show the filename of files included with \lstinputlisting; also try caption instead of title
}




%renomeia o titulo dos tipos figure
\renewcommand{\figurename}{Figura}
%renomeia o titulo dos tipos Table
\renewcommand{\tablename}{Tabela} 
\renewcommand{\lstlistingname}{Código} 
\renewcommand{\chaptername}{Capítulo}
\renewcommand{\contentsname}{Sumário}
\renewcommand{\bibname}{Referências}
\renewcommand{\listfigurename}{Lista de Figuras}
\renewcommand{\listtablename}{Lista de Tabelas}

\graphicspath{{figuras/}{fig_site/}}

\newcommand{\ttbu}[1]{\texttt{\textbf{\underline{#1}}}}

\usepackage{ifthen}

